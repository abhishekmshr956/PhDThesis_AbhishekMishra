%*******************************************************
% Abstract+Sommario
%*******************************************************

\pdfbookmark{Summary}{Summary}
\begingroup
\let\clearpage\relax
\let\cleardoublepage\relax
\let\cleardoublepage\relax

\chapter*{Summary}


Extracting the direction of visual motion is a critical task for the survival of every animal that relies on vision. Although the anatomy of visual circuits was cartographed in mammals and flies alike over a century ago, the physiology of these cells remained speculative for long. Behavioral experiments with the snout weevil \textit{Chlorophanus viridis} served as the basis for the proposal of an algorithmic model for elementary motion detection. In essence, two spatially separated inputs, one of which is delayed in time, are integrated in a non-linear operation. Since the first stage of the visual transduction cascade -- photoreceptors -- do not distinguish the direction of motion, the computation must occur in the circuitry downstream. In the fruit fly \textit{Drosophila melanogaster} dense electron microscopic reconstruction, cell-type-specific driver lines, and genetically encoded indicators have led to significant progress over the past decade. We now know that: (1) motion information is split into parallel ON and OFF pathways, (2) T4 and T5 cells are the first direction-selective neurons in the fly motion vision pathway, (3) and two complementary mechanisms create direction-selectivity on T4 and T5 cells dendrites. One is responsible for enhancing signals when cells are stimulated in their preferred direction and the other for suppressing motion signals that are presented in the opposite direction.

Since the beginning of my doctoral work, the functional roles of the inputs to T4 and T5 cells are under debate. In the first manuscript, we used 2-photon calcium imaging in combination with white noise stimuli to identify the spatio-temporal response properties of all columnar input elements to the elementary motion detectors. After reverse correlating the signals with the stimulus, we found that input elements exhibit a range of temporal properties that can be grouped into two classes: low-pass filter and band-pass filter. Placing the cells onto the model in different spatial configurations and looking for those arrangements which would recapitulate T4 $\&$ T5 responses most faithfully, we were able to make suggestions about the anatomical wiring of the presynaptic partners to the T4 and T5 cells dendrites. This finding was later confirmed by an EM reconstruction data set. We also showed that the filter characteristics of the input elements are not fixed but instead can be influenced by neuromodulation. 

In this study, we exclusively used the genetically encoded calcium indicator GCaMP. The relationship between the calcium signal of any given cell with its transmitter output can however, be complex. In the second study, we, therefore, set out to measure the transmitter release of all glutamatergic neurons in the motion vision pathway. 
We were able to show that spatial aspects of the receptive fields, measured with the recently developed glutamate sensor iGluSnFR, and GCaMP, are preserved. In the temporal domain, however, we find that these responses are substantially sped up in the glutamate signal. The results give a more realistic picture of the dynamics of the glutamatergic input, which the motion-sensitive T4 cell receives from a presynaptic partner (Mi9).

Flies reliably track the direction of motion over a large range of velocities and contrasts. Correlation-type motion detector models, however, are vulnerable to contrast changes when estimating the velocity in natural environments. In the last study, we set out to close this gap. Using 2-photon calcium imaging, we demonstrated that neurons presynaptic to T4 and T5 cells implement an adaptive non-linear gain control mechanism. By blocking the output of medullary neurons, we were able to show that this adaptive gain reduction, at least partially, arises from feedback rather than feedforward inhibition. Lastly, integrating a divisive normalization step into the model dramatically increased its motion vision robustness. 

The three studies included in this thesis are presented chronologically and were published in peer-reviewed journals. 


%-emphasize visual neuroscience
%-then motion vision 
%-then drosophila : genetics, stereotyped behaviors, identified cell types, %connectome, tools, 
%initially only behavior, models and e-phys of large cells. circuit in between %remained blackbox. now direct subcellular measurements of previously %unidentified neurons possible therefore first study...
% second study : transmitter phenotype plus dynamics of release of Mi9.
% 
%Last study : motion detectors are dependent on contrast changes and perform %unreliable to natural images. Behavior is stable. Therefore the circuit found %a way to be better than the model. Gain control needed and indeed found in %early motion vision pathway and then relayed onto elementary motion detectors. %General mechanism in many sensory modalities and species.
%Integration of non-linear adaptive mechanisms compensates for this %discrepancy.


\vfill


\endgroup			

\vfill

