%*******************************************************
% Abstract+Sommario
%*******************************************************

\pdfbookmark{Summary}{Summary}
\begingroup
\let\clearpage\relax
\let\cleardoublepage\relax
\let\cleardoublepage\relax

\chapter*{Summary}


For animals to survive, detecting the direction of image motion is an essential component of visual computation. An individual photoreceptor, however, does not explicitly represent the direction in which the image is shifting. Comparing neighboring photoreceptor signals over time is used to extract directional motion information from the photoreceptor array in the circuit downstream. To implement direction selectivity, two opposing models have been proposed. In both models, one input line is asymmetrically delayed compared to the other, followed by a non-linear interaction between the two input lines. The Hassenstein-Reichardt (HR) model proposes an enhancement in the Preferred Direction (PD): the preferred side signal is delayed and then amplified by multiplying it with the other input signal. In contrast, the Barlow-Levick (BL) detector proposes a Null Direction (ND) suppression, whereby the null side signal is delayed and divided from the other input. The motion information is computed in parallel ON and OFF pathways. T4 (ON) and T5 (OFF) are the first direction-selective neurons found in the ON and in the OFF pathway respectively.

In the first manuscript, we found that both preferred direction enhancement and null direction suppression are implemented on the dendrites of all four subtypes of both T4 and T5 cells to compute the direction of motion. A hybrid model combining both PD enhancement on the preferred side and ND suppression on the null side was proposed. Already at the first stage of calculating motion direction, this combined strategy ensures high degree of direction selectivity. 

Further processing in addition to synaptic mechanisms on the dendrites of T4 cells, can improve the direction selectivity of the T4 cells' output signals. Neurons process and communicate information by transforming membrane voltage into calcium signals, leading to transmitter release. In the signaling cascade, computations can occur at different stages: 1.) dendritic integration and processing of voltage signals. 2.) transformation from voltage to calcium and 3.) between calcium and neurotransmitter release. In the second manuscript, we used in vivo two-photon imaging of genetically encoded voltage and calcium indicators, Arclight and GCaMP6f respectively, to measure responses in Drosophila direction-selective T4 neurons. Comparison between Arclight and GCaMP6f signals revealed calcium signals to have a significantly higher direction selectivity compared to voltage signals. Using these recordings we built a model which transforms T4 voltage responses to calcium responses. The model reproduced experimentally measured calcium responses across different visual stimuli using different temporal filtering steps and a stationary non-linearity. These findings provided a mechanistic underpinning of the voltage-to-calcium transformation and showed how this processing step, in addition to synaptic mechanisms on the dendrites of T4 cells, enhances direction selectivity in the output signal of T4 neurons.

The two manuscripts included in this thesis are presented chronologically. The first manuscript was published in a peer-reviewed journal, while the second manuscript is currently under review and is available as a preprint. 


%-emphasize visual neuroscience
%-then motion vision 
%-then drosophila : genetics, stereotyped behaviors, identified cell types, %connectome, tools, 
%initially only behavior, models and e-phys of large cells. circuit in between %remained blackbox. now direct subcellular measurements of previously %unidentified neurons possible therefore first study...
% second study : transmitter phenotype plus dynamics of release of Mi9.
% 
%Last study : motion detectors are dependent on contrast changes and perform %unreliable to natural images. Behavior is stable. Therefore the circuit found %a way to be better than the model. Gain control needed and indeed found in %early motion vision pathway and then relayed onto elementary motion detectors. %General mechanism in many sensory modalities and species.
%Integration of non-linear adaptive mechanisms compensates for this %discrepancy.


\vfill


\endgroup			

\vfill

