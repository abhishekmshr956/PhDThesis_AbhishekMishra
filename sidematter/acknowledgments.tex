%*******************************************************
% Acknowledgements
%*******************************************************
\pdfbookmark{Acknowledgements}{Acknowledgements}

\begingroup
\let\clearpage\relax
\let\cleardoublepage\relax
\let\cleardoublepage\relax

\chapter*{Acknowledgements}

First I want to thank Axel for being the supervisor I needed in order to achieve my PhD. He supported me in times of scientific and personal desperation alike and at the same time accompanied the overwhelming majority of good days with splendid happy hours, sunny days in the Würmbad or in the Biergarten. He always found the right balance of giving me enough freedom to develop my scientific ideas and projects but then also gave me pushes in the right direction whenever I needed them. Without him I most likely would have given up at some point and would not have completed my PhD. I thank the whole GSN (graduate school) for organizing many scientific and non-scientific events and meetings. I also want to thank the rest of my TAC, Ruben Portugues and Laura Busse for helpful advice and stimulating discussions concerning my projects.

No less important were the colleagues in the lab who especially in the early phase taught my invaluable lessons about science and included me in fruitful projects. Although the whole lab contributed to an atmosphere that was and still is highly motivating and collaborative, I need to highlight some of my fellow colleagues. I want to thank all post and current members of the P7 office, especially Aljoscha Leonhardt, Michael Drews, Anna Schützenberger, Nadya Pirogova, Birte Zuidinga. I am in deep gratitude for correcting the most bluntly stupid programming errors I made, last-minute correcting abstracts or posters that were close to submission deadline or helping me trouble-shoot the stimulus arena or my 2-photon setup. But above all I want to thank them for being amazing colleagues and friends that made me actually enjoy a great deal of my work and leisure in the past 5 years or so.

Furthermore I want to thank Georg Ammer who really is a walking science dictionary. No matter how complex the question is he is able to shoot out reliable facts about almost every scientific topic regardless of how far away it is from motion vision at speeds Google can only dream of. Topics of his expert knowledge include birds, cephalopodes and the history of shipping just to name a few dubious ones. I also need to thank him for dozens of moutaineering tours where we hiked, scrambled, climbed, skied and snowboarded up and down probably more meters than there are neurons in a fruit fly's brain. 

I thank Matthias Meier and Étienne Serbe for countless good hours inside and outside the lab, summers long of trying to break the world record in the art of "danteln". They kindly integrated me already into the group when others were still hesitant and wanted to wait if the guy from Franconia is trust-worthy. I also want to thank É.T. especially for letting me move in with him when I fell victim to the notorious housing market in Munich. But also special thanks to MM for numerous climbing trips, hundreds of table tennis games and even more beers we had together. Needless to say both of course are also brilliant scientists who's opinion I value a lot. 

I thank Matthias Gumbert (Gumbi) for sharing my pain during the whole process of writing this thesis. I also thank him for rappeling me down to safety when I got myself in a precarious situation during our first and last mountain tour. 

I thank Niko Hörmann for countless espressi, cakes and interesting discussions about many topics, including scientific ones. He always has a different view on things, which, after rejecting his hypothesis, makes you actually relfect your own opinion. 

I thank Amalia Braun, for being the engine of good will to make pandemic times as social as possible and always motivating the people in the lab to not go straight home, but rather sit together and have an after work beer, which usually reduces the grumpiness immensely. 

I thank Jürgen Haag alias Bulle for helping me build my 2-photon microscope and trouble shooting the reoccurring problems with the sometimes moody setup. 

I also thank Sandra Fendl for great and uncomplicated collaboration in the glutamate imaging project, unlimited patience during the PDE project and fun times at the Oktober or Starkbierfest.

Last but by no means least I want to thank my girlfriend Eva Luca Hetzel and my family. Eva supported me through all phases. She shared my excitement when I came home with interesting data, solaced me when the data turned out to be in fact not so interesting or gave me space whenever I needed quiescence. I'd like to thank my mother for supporting me as long as she could, especially in school times, paying hours and hours of math tutoring lessons despite living herself on a low budget. She supported me when teachers, friends, and sometimes even myself gave up on me. She got me through gymnasium and made me the first in the family ever receiving the Abitur, studying and now finally finishing my PhD. I will never forget the sacrifices she made for me and I will be thankful until the rest of my life. 



\endgroup



