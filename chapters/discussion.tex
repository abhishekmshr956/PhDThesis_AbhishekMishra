%************************************************
\chapter{Discussion}
\label{chp:Discussion}
%************************************************
\section{Effect of voltage to calcium transformation on T4 output signals}
In addition to the dendritic integration of postsynaptic voltages or after the action potential is generated, further computations can occur in the transformation between voltage and calcium, or between calcium and neurotransmitter release. Neuronal signaling and information processing involves the transformation of membrane voltage into calcium signals, which lead to transmitter release. In our second manuscript \ref{sct:manuscript_mishra_haag}, we showed that voltage to calcium transformation in T4c neurons enhances their direction selectivity. The calcium signals in T4c cells had a significantly higher direction selectivity and tuning compared to membrane voltage across different stimuli conditions. The direction selectivity index for calcium signals compared with voltage signals for a few stimuli conditions was previously found to be higher in a study in T5 cells using ASAP2f as an optical voltage indicator \parencite{Wienecke2018}. How does this affects the output signal of T4/T5 neurons?

As calcium is required for neurotransmitter release, this is expected to increase the direction selectivity of T4/T5 cells' output signals. In the lobula plate, T4/T5 cells provide inputs onto large lobula plate tangential cells that are depolarized during preferred and hyperpolarized during null direction motion \parencite{Mauss2014}. For example, vertical system (VS) cells with dendrites in layer 4 receive direct excitatory inputs from downward tuned T4d/T5d neurons causing depolarization during motion in the downward preferred direction. These VS cells also receive indirect inhibitory inputs from upward tuned T4c/T5c neurons via glutamatergic LPi3-4 neurons projecting from layer 3 to layer 4 causing hyperpolarization in VS cells during motion in the upward null direction. Upon silencing LPi3-4 neurons’ synaptic output via tetanus toxin, VS neurons depolarization response in the preferred direction did not change, but the null direction response was absent \parencite{Mauss2015}. This suggests T4/T5 do not release any transmitter in response to null direction motion, which matches our findings for the calcium responses. Thus, voltage to calcium transformation increases direction selectivity in T4/T5 cells and this enhances direction selectivity in downstream neurons. 

\section{Differential expression of voltage-gated calcium channels}

In manuscript \ref{sct:manuscript_mishra_haag}, we built a model to capture voltage to calcium transformation in T4c, Mi1, and Tm3 cells. A simple model with a single low-pass filter was able to reproduce calcium responses in non-direction-selective Mi1 and Tm3 cells, whereas a more complex model combining the output of two low-pass filters via a multiplication was required to reproduce T4c calcium responses. The direction selectivity for the simple model signals for T4c was lower compared to the multiplicative model. This suggests that voltage-calcium transformation in Mi1 and Tm3 cells is different from those in T4c cells. 

Differential expression of voltage-gated calcium channels in these cells could explain the different voltage to calcium transformation. Voltage-gated calcium channels mediate depolarization-induced calcium influx that drives the release of neurotransmitters. The $\alpha1$-subunit of the voltage-gated calcium channels forms the ion-conducting pore, which makes it distinct from other calcium channels. Three families of genes encode $\alpha1$ subunits. \textit{Drosophila} genome has one $\alpha1$ subunit gene in each family: $\alpha1D$ ($Ca_{v}1$), cac ($Ca_{v}2$), and $\alpha1T$ ($Ca_{v}3$) \parencite{Littleton2000, King2007}. In \textit{Drosophila} antennal lobe projection neurons, cac ($Ca_{v}2$) type and $\alpha1T$ ($Ca_{v}3$) type voltage-gated calcium channels are involved in sustained and transient calcium currents, respectively \parencite{Gu2009, Iniguez2013}. According to a RNA-sequencing study \parencite{Davis2020}, $\alpha1T$ ($Ca_{v}3$) mRNA have higher expression in Mi1 ($2050.16$ Transcripts per Million (TPM)) compared to T4 ($686.68$ TPM) and Tm3 ($336.45$ TPM). While cac ($Ca_{v}2$) mRNA have higher expression in T4 ($1298.53$ TPM) compared to Mi1 ($986.25$ TPM) and Tm3 ($817.61$ TPM). Different expression of voltage-gated calcium channels could cause different voltage to calcium transformations in non-direction selective and direction-selective cells.

\section{Comparison between fly and mammalian motion vision}
