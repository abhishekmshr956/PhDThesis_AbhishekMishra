%************************************************
\chapter{Introduction}
\label{chp:Introduction}
%************************************************

\section{Motion Vision}
what is motion vision. what is the central question we want to answer ? Why do we try to answer above question using drosophila ?

\section{\protect\NoCaseChange{\textit{Drosophila}} as a model organism}
\textit{Drosophila melanogaster} is one of the most powerful model organisms available for functional dissection of neural circuits. It allows for most sophisticated in vivo neural manipulations - imaging, activation \& suppression of neural activity. Over 100 years of research in \textit{Drosophila} has allowed generation of thousands of fly 'driver-lines' which can be used to express genes of interest in neuron-specific manner \parencite{Pfeiffer2008}. Along with this, \textit{Drosophila} allows several practical working advantages : They are small, has a short generation time of about 10 days \& easy to grow in a lab. 

\textit{Drosophila} brain is estimated to contain about 100,000 neurons \parencite{Zheng2018}. \textit{Drosophila} brain involves computation of modest complexity. These computations are implemented in circuits that contain a limited number of neurons, and with \textit{Drosophila} genetic armoury almost each of these neuron can be precisely targeted. However even with comparatively less complexity, there are surprising parallels between how fly and mammalian brains process information \parencite{Borst2015}. Insights about the nervous system obtained in \textit{Drosophila} is thus often relevant for other species \parencite{Bellen2010, Venken2011}.

\section{Tools for functional dissection of \protect\NoCaseChange{\textit{Drosophila}} neural circuits}
To have a detailed understanding of how a neural circuit functions, we need to know role each individual neuron plays in that particular circuit. To achieve this, we would like to perform following three types of manipulations on the given neuron : (i) record neuronal activity from the neuron, (ii) activate the neuron \& (iii) silence the neuron. Fortunately years of research in Drosophila has provided us with multiple tools in order to be able to perform these manipulations in the choice of neuron we want. The most important tool which enables us to do this in neuron-specific manner is the Gal4-UAS system. 

\subsection{GAL4-UAS / LexA-lexAop}
Following the discovery of transposable DNA sequences (P-elements) in the \textit{Drosophila} genome\parencite{Rubin1982}, \cite{Brand1993} designed GAL4-UAS system. The GAL4-UAS system is the workhorse of \textit{Drosophila} genetics. The GAL4-UAS system is a binary expression system consisting of two main components : the yeast transcriptional factor GAL4 expressed in a specific pattern and a reporter gene under the control of UAS promoter that is silent in the absence of GAL4. Gal4-UAS system essentially involves crossing two fly lines : one called the 'driver-line', defines which neurons express required effector gene; the other called 'reporter-line', defines what gene is expressed in the neurons defined by driver line [see Figure  \ref{fig:gal4uas}]. 

Another independent binary transcriptional system which can be used is LexA-lexAop system. This method is based on the bacterial DNA-binding operator lexAop and controlled by the expression of LexA. LexA binds to and activates the lexA operator(lexAop). We can use LexA-lexAop system in combination with GAL4-UAS system to simultaneously express gene of interest in two different neuronal population. Using combination of GAL4-UAS and LexA-lexAop system, we can express following three types of reporter genes : Indicator, Suppressor \& Activator.
\begin{figure}[h]
\centering
\includegraphics[scale=0.8]{borstgal4uas}
\caption{Gal4-UAS}
\label{fig:gal4uas}
\end{figure}
\subsection{Calcium Indicator : GCaMP for recording changes in intracellular calcium}
To record neuronal activity, one can use GAL4-UAS system to express GFP in the required neuron, and then use somatic patch recording to record neural activity from the neuron \parencite{Wilson2004, Joesch2008}. However, neurons in the optic lobe of \textit{Drosophila} are often too small in size for successful electrophysiological recording. To overcome this we can use calcium indicators as proxy for neuronal activity. 

Neural activity causes rapid changes in intracellular free calcium\parencite{Baker1971, Sabatini2002, Egelhaaf1995}. We can thus express genetically encoded calcium indicators (GECIs) which changes its fluorescence according to the change in concentration of intracellular calcium. GECIs typically consist of a calcium-binding domain - calmodulin, calmodulin-binding peptide M13, and a reporter element which is based on either a single fluorescent protein or two fluorescent proteins \parencite{Broussard2014}. In the case of single fluorescent protein for example in GCaMPs, calmodulin (CaM) binds the M13 peptide in the presence of calcium. This coupling results in conformational changes in the fluorescent protein, resulting in change in fluorescence intensity \parencite{Nagai2001}. In the case of two fluorescent proteins, conformational changes lead to Fluorescence Resonance Energy Transfer (FRET) between two fluorescent proteins with overlapping excitation and emission spectra \parencite{Miyawaki1997}. In this thesis, we use GCaMP6 \parencite{Chen2013} in combination with two-photon microscopy for recording neural activity.

\subsection{Voltage Indicator : Arclight for recording changes in neuronal membrane potential}
In order to measure the membrane potential changes in neurons, electrophysiology has been the most frequently used method. However, as mentioned earlier, due to the small size of neurons in the optic lobe, single-cell electrophysiological recordings of these neurons have been difficult. Genetically encoded voltage indicators (GEVIs) have evolved as powerful tools for recording changes in neuronal membrane potentials. Optical methods of monitoring brain activity are appealing because they allow simultaneous, noninvasive monitoring of activity in many individual neurons and different parts of the brain. 

We use a fluorescence protein (FP) voltage sensor called Arclight \parencite{Jin2012}. Arclight is based on the fusion of voltage sensing domain of \textit{Ciona intestinalis} voltage sensitive phosphatase \parencite{Murata2005} and the fluorescent protein super ecliptic pHluorin with an A227D mutation. Arclight's fluorescence decreases with membrane depolarization and increases with membrane hyperpolarization. We used Arclight in combination with two-photon imaging to record changes in neuronal membrane potential.

\subsection{Suppressor : $shibire^{ts}$, GtACR for silencing neuron}
In order to understand the principles of information processing in a neural circuit, along with recording neural activity we would also like to either suppress or activate neural activity in other neurons in the circuit. Inactivation and activation of genetically defined cell types help establish causal relations in specific group of neurons and neural circuit. There are several tools which allow for inactivation of a neuron. Cell death genes such as \textit{reaper (rpr)} and \textit{head involution defective (hid)} or \textit{grim} induce apoptosis \parencite{Chen1996, Grether1995}.  The tetanus toxin light chain (TeTxLc) cleaves the synaptic vesicle protein synaptobrevin, and inhibits neurotransmitter exocytosis at chemical synapses \parencite{Sweeney1995}. The expression of Kir - an inwardly rectified potassium channel causes neurons to hyperpolarize, resulting in suppressed excitability \parencite{Johns1999}. While using Gal4-UAS system to express these effectors provide effective control over the functionality of the targeted neurons, it also creates some unwanted problems. First, because most enhancers are active at several stages of development, it is difficult to avoid the toxic effects of expressed gene product on development. Second, if particular neurons are eliminated during development, their loss in function may be compensated by some other neuron, hence making interpretation of results difficult. third, it is not possible to observe acute effect elicited by the temporal inactivation of particular neurons. To overcome these limitations, we can use conditional effector proteins like $shibire^{ts}$ \& GtACR, which is activated by higher temperature \& light respectively. 

\textit{Drosophila shibire} encodes the protein dynamin, which is involved in the process of endocytosis and is essential for vesicle recycling. The dominant-negative temperature sensitive allele $shibire^{ts}$ is defective in synaptic vesicle recycling at restrictive temperature ($>29\degree C$) which results in rapid and reversible inhibition of synaptic transmission \parencite{Toshihiro2001}. In \textit{Drosophila}, \cite{Joesch2010} showed that flies expressing $shibire^{ts}$ if exposed to persistent heat-shock for one hour at restrictive temperature $(37\degree C)$, the output of affected cells is suppressed for several hours. This experimental method gives us a longer time duration, which allows us to record neural activity from the fly while the activity from the neuron expressing $shibire^{ts}$ is suppressed. However, we get this extra time duration at the cost of losing reversibility of activity.

The tools we have mentioned above allows for a cell-type-specific neuromodulation. However, in addition to the cell-type-specificity we would also like to have temporally accurate and reversible neuromodulation. \textit{Guillardia theta} Anion Channel Rhodopsins (GtACR1 and GtACR2) can provide us with these additional advantages \parencite{Govorunova2015}. GtACRs impart strong light-gated chloride conductance and is much more light-sensitive than Halorhodopsin class of chloride pumps. In particular with respect to the fly visual visual system, we used GtACR1 since its activation spectrum is shifted towards longer wavelengths with respect to five of the six \textit{Drosophila} rhodopsins (except rhodopsin 6) \parencite{Mauss2017, Mohammad2017}. Thus, we can use Gal4-UAS system to express GtACR in neurons we want to hyperpolarize \& silence the neural activity. Simultaneously, we can express GCaMP in downstream neurons using LexA-lexAop system \& record their neural activity, while silencing the neurons expressing GtACR. 

 
\subsection{Activator : ReaChr for activating neuron}
Along with using GCaMP for imaging, $shibire^{ts}$ or GtACR for suppressing neural activity, we would also like to have a tool for activating neuron. Ideal method for activation requires excellent temporal control. Light-gated cation channels - Channelrhodopsins(ChRs) can be used for this purpose. ChRs are light-gated, non-specific cation channels that allow selective depolarization of genetically targeted cells. Here we have used red-activatable ChR (ReaChr), a variant of ChR \parencite{Lin2013, Busch2018}. 

Therefore by combining the use of GCaMP for calcium imaging and GtACR for suppression or ReaChR for activation, we can better understand the functions of neurons present in a given neural circuit.


\section{Fly motion vision system}
\textit{Drosophila} visual processing pathway is comprised of retina, lamina, medulla, lobula \& lobula plate, each arranged in columnar, retinotopic fashion (Fig.) \parencite{Fischbach1989}. If one were to record from a single photoreceptor in the retina, it would show similar response to a patch of moving image irrespective of the direction of motion : meaning it is not direction-selective. However, if one records around 4 synapses downstream, Lobula Plate Tangential Cells (LPTCs) depolarise in response to a patch of image moving in its preferred direction and hyperpolarize if the image moves in opposite direction or the null direction. HS(Horizontal System) cells for example are responsive to horizontal motion \parencite{Schnell2010}, while VS(Vertical System) cells are responsive to vertical motion\parencite{Joesch2008}(Fig.). LPTCs, however integrate over large parts of visual fields, i.e. they are not local motion detectors. Hence, the natural question arises : which cells are the local motion detectors ? 

The answer to the above question is : T4 \& T5 are first local motion detectors found in the \textit{Drosophila} ON \& OFF motion vision pathway respectively. Four sub-population of T4a-d \& T5a-d cells tuned to the four cardinal directions and projecting to the four layers in the lobula plate can be found within each column \parencite{Maisak2013}. This leads to next question : what makes T4 \& T5 direction selective ? To answer this question, we need to investigate the cells which are present in between the non-direction-selective photoreceptors in the retina \& direction-selective T4, T5 cells in Medulla \& Lobula respectively. The columnar cell types of the lamina, medulla, lobula anad lobula plate have all been identified and described \parencite{Fischbach1989, RamonyCajal1915}.  

Lamina is organized in an array of $\sim 750$ retinotopic columns (also called 'cartridges'). Each column corresponds to $\sim 5\degree$ discrete sample of the visual world. Light sensitive photoreceptors, R1-6 project their axons into each lamina column. Two other photoreceptors, R7 \& R8 pass thorugh lamina and synapse in specific layers of medulla. Along with photoreceptor axons, lamina includes 5 lamina output neurons (L1-L5), six putative feedback neurons and one lamina intrinsic neuron. Lamina columnar monopolar neurons, L1-L5 send their axonal projections into specific layers of the medulla. \parencite{Fischbach1989, Tuthill2013}. 

Medulla consists of more than 60 different cell types in a single column. They can be clustered into different groups based on their anatomy. Medulla intrinsic ('Mi') neurons connect different layers of the medulla to each other. Trans-medulla ('Tm') neurons connects specific layers of medulla to various layers in the lobula. Trans-medulla Y ('TmY') neurons connect specific layers of the medulla to various layers in the lobula and lobula plate. Directions-selective cell T4 connects medulla layer 10 to the four layers of lobula plate.

While the circuit was known, the small size of these neurons made electrophysiological recordings difficult. Only after the advent of modern 2-photon imaging in combination with using Gal4-UAS system to express GcaMP in these cells, it became possible to record neuronal activity from these cells. Experiments over the years using these techniques revealed following interesting results : (a) Visual processing in \textit{Drosophila} occur in two parallel processing pathways for brightness increment (ON) \& brightness decrement (OFF) \parencite{Joesch2010, Joesch2013, Strother2014, Eichner2011, Behnia2014} (b) T4 \& T5 are first local motion detectors found in the \textit{Drosophila} ON \& OFF motion vision pathway respectively. Four sub-population of T4a-d \& T5a-d cells tuned to the four cardinal directions and projecting to the four layers in the lobula plate can be found within each column \parencite{Maisak2013}. 
 
\begin{figure}[h]
\centering
\includegraphics[scale=0.3]{flyopticlobe1}
\caption{flyopticlobe1}
\label{fig:flyopticlobe1}
\end{figure}

\begin{figure}[h]
\centering
\includegraphics[scale=0.8]{flyopticlobe}
\caption{flyopticlobe}
\label{fig:flyopticlobe}
\end{figure}

\subsection{Parallel ON \& OFF processing pathways}
In striking similarity to mammalian retina \parencite{Masland2012}, visual processing in \textit{Drosophila} occurs in two parallel ON \& OFF processing pathways \parencite{Borst2015}. The ON pathway transmits information about brightness increments, while the OFF pathway transmits information about brightness decrements. In order to understand split of photoreceptor (R1-R6) signals into ON \& OFF pathway, studies have been done blocking Laminar monopolar cells, while simultaneously recording from downstream LPTCs neuron. While blocking L1 neurons resulted in specific reduction of LPTC's response to ON stimulus (brightness increment), blocking L2 neurons resulted in specific reduction of LPTC's response to OFF stimulus (birghtness decrement) \parencite{Joesch2010}. To answer how does blocking L1 \& L2 affect response at the level of T4 \& T5, in \textbf{manuscript 2} we silenced L1 using GtACR and simultaneoulsy recorded neural activity from T4 using GCaMP. Similarly, we also silenced L2 and recorded simultaneously from T5. Silencing L1 resulted in loss of response of T4 cells to ON stimulus and silencing L2 resulted in loss of response of T5 cells to OFF stimuli. These experiments together suggests that L1 pathway specifically transmits information about brightness increments to downstream ON motion detector T4, and L2 pathway specifically transmits information about brightness decrements to downstream OFF motion detector T5. 

\subsection{T4 \& T5 cells}
Based on previous studies from \parencite{Fischbach1989, Buchner1984}, T4 \& T5 were long thought to be the prime candidates for local motion detectors in the ON \& OFF pathway respectively. However, due to its small size it was difficult to do electrophysiological recordings from T4 \& T5 cells. This problem was solved using combination of 2-photon imaging and Gal4-UAS system to express GCaMP in T4, T5 cells to record its neural activity in response to the ON \& OFF stimuli. Stimulating the flies in four cardinal directions (front-back, back-front, upwards and downwards), \cite{Maisak2013} recorded direction selective activity from T4/T5 cells. Four sub-population of T4a-d \& T5a-d cells tuned to the four cardinal directions and projecting to the four layers in the lobula plate were found within each column. Further, T4 were found to respond specifically to ON stimulus \& T5 were found to respond specificaly to OFF stimulus. Blocking T4 and T5 cells led to a complete loss of motion response in lobula plate tangential cells \parencite{Schnell2012}, of the optomotor response of tethered walking flies \parencite{Bahl2013}. Specific blocking of T4 cells led to reduction in LPTC \& optomotor responses to ON stimulus selectively. While specific blocking of T5 cells led to reduction in LPTC \& optomotor responses to OFF stimulus selectively. These results together suggest T4 \& T5 to be the elementary motion detector for ON \& OFF pathway respectively\parencite{Maisak2013}.      

\section{Neural circuit underlying direction selectivity}
Having identified T4 \& T5 as the elementary local motion detectors, the next question which arises is which cells provide synaptic inputs to T4 \& T5 cells. Electron Microscopy (EM) studies \parencite{Shinomiya2019, Takemura2017} provided answer to this question. \cite{Shinomiya2019} used FIB-SEM (Focused Ion Beam Serial Electron Microscope) to record a volume of the optic lobe comprising of seven columns of the medulla, lobula and lobula plate. They identified all the different neuron types providing inputs to the T4 \& T5 cells. T4 cells receive input from Mi1, Tm3, Mi4, Mi9, C3, CT1 and TmY15. T5 cells receive input from Tm1, Tm2, Tm4, Tm9, CT1, TmY15, LT33 and Tm23. T4 \& T5 cells' dendrites span over several columns along the preferred direction of the motion. The authors could also locate where the different cell type synapse onto the dendrites of T4 \& T5. For example for T4c with preferred direction of motion as upwards receives input from Mi1, Tm3 \& TmY15 in the central part of its dendrite, from Mi9 and T4c on the ventral part and from Mi4, C3 and CT1 on the dorsal part of its dendrite [figure \ref{fig:t4t5inputsynapses} top]. For T4d with preferred direction as downwards, it receives input from Mi1, Tm3 \& TmY15 in the central part, from Mi9 \& T4d on the dorsal part and from Mi4, C3, CT1 on the ventral part of its dendrite. In summary, all T4 subtypes receive inputs from Mi1, Tm3 \& TmY15 in the central part, from Mi9 on the preferred side (i.e. the side from which a preferred direction stimulus approaches) \& from Mi4, C3 \& CT1 on the null side (i.e. the side from which a null direction stimulus approaches) of their dendrite. Similarly, all T5 subtypes receive inputs from Tm1, Tm2 and Tm4 on the central part, Tm9 on the preferred side \& CT1 on the null side of their dendrite [figure \ref{fig:t4t5inputsynapses}]. %a figure showing summary of different inputs.    

\begin{figure}[h]
\centering
\includegraphics[scale=0.2]{t4t5inputsynapses}
\caption{t4t5inputsynapses}
\label{fig:t4t5inputsynapses}
\end{figure}

Most of these input elements have been characterised physiologically \parencite{Arenz2017, Serbe2016, Strother2017, Meier2019, Borst2020}. None of these cells were found to be direction-selective. Hence, we can conclude that T4 \& T5 are the first elementary motion detector found in the ON \& OFF pathway respectively, and thus represents an important processing stage where direction is computed.    

\section{Neural algorithm underlying direction selectivity}
The underlying mechanisms generating direction selectivity and the neural correlates implementing these mechanisms have not been yet fully understood. Classically two opposing models have been proposed for implementation of direction selectivity. Both these model use two input lines, where one of the input line has been asymmetrically delayed compared to the the other, which is then followed by a non-linear interaction. The Hassenstein-Reichardt (HR) model proposes a Preferred Direction (PD) enhancement : the signal on the preferred side is delayed and is subsequently amplified using multiplication of the signal from the other input line [\textbf{Figure}]\parencite{Hassenstein1956}. The Barlow-Levick (BL) detector however, proposes a Null Direction (ND) suppression : the signal on the null side is delayed and divides the signal from the other input resulting in suppression \parencite{Barlow1965} . \cite{Haag2016a} used apparent motion stimuli to show that both the mechanisms i.e. PD enhancement \& ND suppression is used by T4 \& T5 cell to produce a direction selective response. In our \textbf{manuscript1} - \parencite{Haag2017}, we showed that all four subtypes of T4 \& T5 indeed uses both PD enhancement \& ND suppression to produce direction selective responses. Therefore, a new model combining both PD enhancement on the preferred side \& ND suppression on the null side was proposed. The next important task now is to identify neural correlates implementing these mechanisms. 

The model requires a fast input at the center, slow input providing excitation on the preferred side \& slow input providing suppression on the null side. Interestingly, from the anatomical \& functional characterization of input data we have discussed earlier, we could predict the input neurons for T4 providing these three kind of inputs. Mi1 is a fast neuron providing input at the central part of dendrite, thus a candidate for central fast input. Mi9 is a slow neuron providing input on the preferred side of the dendrite, hence a candidate for slow excitatory input. Mi4, C3 \& CT1 are slow neurons providing input on the null side of the dendrite. In \textbf{manuscript2} we have expressed GtACR or ReaChr to silence or activate input neurons respectively. Simultaneously, we express GCaMP in the downstream T4c/T5c neurons. These experiments together help us understand the functional contribution of each of these input neurons to direction selective signals in T4.




